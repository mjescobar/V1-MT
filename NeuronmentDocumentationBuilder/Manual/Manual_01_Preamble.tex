% Márgenes
\usepackage[left=4cm,right=3cm,top=3cm,bottom=3cm]{geometry}
% Acentos
\usepackage[utf8x]{inputenc}
% To indent first paragraf of *section
\usepackage{indentfirst}
% Fórmulas
\usepackage{amssymb, amsmath, amsmath}
% Imágenes
\usepackage{graphicx}
% Multiples Columnas
\usepackage{multicol}
% For multiple possible colors
\usepackage[usenames,dvipsnames]{xcolor}
% Para columnas alineadas de ancho fijo
\usepackage{array}
\newcolumntype{L}[1]{>{\raggedright\let\newline\\\arraybackslash\hspace{0pt}}m{#1}}
\newcolumntype{C}[1]{>{\centering\let\newline\\\arraybackslash\hspace{0pt}}m{#1}}
\newcolumntype{R}[1]{>{\raggedleft\let\newline\\\arraybackslash\hspace{0pt}}m{#1}}
% Vínculos
\usepackage{hyperref}
\hypersetup{
  linktocpage,
  colorlinks=true,
  pdfborder={0 0 0},
  allcolors=Sepia
% linkcolor
% anchorcolor
% citecolor
% filecolor
% menucolor
% runcolor
% urlcolor
}
% Posicionamiento de vínculos a imágenes
\usepackage[all]{hypcap}
% Glosario de Términos
\usepackage[sanitize=none,sanitize={sort=false}]{glossaries}
\newglossarystyle{mylist}{%
  \renewenvironment{theglossary}{\begin{multicols}{2}\begin{list}{}{\leftmargin=1em \itemindent=-1em}}{\end{list}\end{multicols}}
  \renewcommand{\glsgroupheading}[1]{\item {\textbf ##1}}%
  \renewcommand{\glossaryentryfield}[5]{%
    \item
    \textbf{\glstarget{##1}{##2}: ##4}
    
    \space ##3 \\
    \space [##5]
  }
}
\glossarystyle{mylist}
\makeglossaries
\setglossarystyle{mylist}
\makeglossaries
% Para la numeración de las subsubsecciones
\setcounter{secnumdepth}{4}
% Para agregar las subsubsecciones al índice
\setcounter{tocdepth}{1}
% Para código
\usepackage{listings}
% Para tablas largas
\usepackage{longtable}
% Para rellenos de lorem ipsum
\usepackage{blindtext}
% Forzar dump de figuras
\usepackage{placeins}

% Labels for pending editions
\def \lotharcl {{\color{red} CHECK LINKS}}
\def \lotharpi {{\color{red} PENDING IMPLEMENTATION}}

% Labels for common glossary terms
\def \NPROC {\glssymbol{neuronmentprocedure} }

% Automatic linking for message and sections
\newcommand{\MyRef}[1]{{\textbf{#1}}}

%Fuente Times
%\usepackage{mathptmx}
%Hyphenation en español
%\usepackage[spanish]{babel}
% Para el tamaño de fuentes específicos
%\usepackage{anyfontsize}

%% Dividir direcciones web
%\usepackage{breakurl}
%% Operaciones con strings
%\usepackage{xstring}
%% Texto en color
%\usepackage{color}
%\DeclareGraphicsExtensions{.eps,.png,.jpg,.pdf,.mps,.gif,.bmp}
%% Vínculos
%\usepackage{url}
%% Multirow
%\usepackage{multirow}
%% Para celdas especiales
%\usepackage{array}