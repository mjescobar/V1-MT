\chapter{Introduction}
\label{ch:introduction}

\section{Purpose}

This document is the Reference Manual for the software Neuronment in his version 1.0 and is intended to provide
all the information required for a full usage of its features.

\section{Style convention}

This document uses the following styles:
\begin{itemize}
  \item \textbf{bold}\\  
  Words in bold are used for commands or situations in which they need to be used specifically as indicated including case type.
  \item \textit{italic}\\  
  Words in italic correspond to words that need to be substituted before using.
  \item \texttt{monospaced}\\  
  Used for examples.
  \item \texttt{<name>}\\  
  In the examples is used for parts that need to be substituted with the real value. Equivalent to italic in normal text.
  \item \texttt{[option]}\\
  In the examples is used to identify optional arguments.
\end{itemize}

Bold and italic may also be used to highlight words.

\section{Problem reporting}

If you find a problem, inconsistency or ambiguous explanation please contact the author at \href{mailto:pedrotoledocorrea@gmail.com}{pedrotoledocorrea@gmail.com}.

\section{How to read this document}

This document is divided in self explanatory chapters presented in 2 groups. Chapters \ref{ch:usersguide}, \ref{ch:instructions} and \ref{ch:messages} refer to the common application environment and the following chapters have the instructions and details for the different possible neurological simulations and training procedures.

\newpage