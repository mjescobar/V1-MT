\chapter{User's Guide}
\label{ch:usersguide}

\section{Overview}

The Neuronment project is a software intended for discreet simulation and training of complex neural networks for neuroscience studies. Its name comes from the words ``Neurological'' and ``Environment'' as the intention is to create a context where different neurological structures can be model, simulated and trained from a simple procedure description file, abstracting all the computational complexities required for the implementation.

Neuronment works by reading a Procedure Description file (standardized extension \texttt{*.nproc}) which should describe the list of all the \gls{neuralnetwork} descriptor parameters and the simulations and/or trainings intended to be calculated. This file, also called  \gls{neuronmentprocedure} (\glssymbol{neuronmentprocedure}), should comply with the \gls{neuronmentsequencersyntax} (\glssymbol{neuronmentsequencersyntax}) in order to be correctly interpreted by the \gls{neuronmentsequencer} (\glssymbol{neuronmentsequencer}).

The \glssymbol{neuronmentsequencersyntax} has been developed with the intention to cover all the possible use scenarios for the intended purposes of the Neuronment project; nevertheless, it is susceptible to changes in future versions that may not be backwards compatible.

The Neuronment project has been build mainly on the experience acquire on the development of the thesis work of Pedro F. Toledo\cite{thesispedro}.

\section{How to Execute Neuronment}

To execute this program it is required the program executable and a ``Neuronment Procedure'' file. To run it you must execute the following line on the shell:

\begin{verbatim}
Neuronment -nproc <file.nproc> [-verbose_messages|-no_verbose_messages]
\end{verbatim}

\begin{itemize}
  \item \texttt{Neuronment}:\\
  Name of the Neuronment executable to use.
  \item \texttt{-nproc <file.nproc>}:\\
  The flag \textbf{-nproc} is used to identify the \glssymbol{neuronmentprocedure} file that should be read by the \gls{neuronmentsequencer}.
  \item \texttt{verbose\_messages no\_verbose\_messages}:\\
  This is an optional setting to force the apparition or not apparition of explanatory texts for the coded messages returned by the \gls{neuronmentsequencer}. You can check the default behaviour by checking the private define \textbf{DEFAULT\_MESSAGES}. {\color{red} CHECK LINKS}
\end{itemize}

\section{Neuronment sequencer}

The neuronment sequencer is the module inside Neuronment on charge of interpreting the \glssymbol{neuronmentprocedure} file specified at the program call.

\subsection{Variable Groups}

\subsection{Message system}
\subsection{Assertions}

\section{Neuronment sequencer syntax}

The following section describes the different aspects of the \gls{neuronmentsequencersyntax} required to create a \glssymbol{neuronmentprocedure} file.

\subsection{Basic rules}
An \glssymbol{neuronmentprocedure} file should be written in ascii and it will be divided in lines using the line breaks as line termination. The resulting lines will be interpreted according their content.

\subsubsection{Empty line}
An empty line prints an empty line to the standard output.

\subsubsection{Comments}
The character \texttt{\#} divides a line between a command string (everything to the left) and a comment string (everything to the right).

If the command string is only composed of non-script characters, it will be considered empty and the whole line will be interpreted as a comment.

If the whole line is a comment, it will be printed out to the standard output, otherwise, the command will be executed and no comment will be printed to the standard output.

\subsubsection{Redirections}
The character \texttt{>} can be used to redirect a command result to a file instead of the standard output.

The text at the right of the \texttt{>} character should be only one valid file name without spaces. After the file name it is possible to add a comment as indicated previously.

\subsubsection{Variables}
All interaction related to parameters for the Neuronment operations and the Neuronment environment are managed by variables which value should be set prior the command execution.

Every variable except a ``Variable Group'' has a parent, by other side every variable could have a child. To specify a variable you should specify the variable group and the variable name separated by a character \texttt{:} as the following example:

\begin{verbatim}
VariableGroup1:Variable1
VariableGroup2:Variable1:ChildVariable1
\end{verbatim}

A variable name must be composed exclusively by letters from a to z, A to Z, numbers and the underscore character.

\subsubsection{Substitutions}
If in order to use a command you would like to use a variable value as parameter instead of a hard coded string, you can substitute the name of the variable for its value by using the character \texttt{\$} just before the variable group as in the following example:

\begin{verbatim}
$VariableGroup1:Variable1
$VariableGroup2:Variable1:ChildVariable1
\end{verbatim}

\subsubsection{Command results}
If a command is intended to return a value after its execution, it will return the values through the Env variable group as it will be indicated in the command specifications.

If you want to store a result you should save it on another variable by using substitution.



















